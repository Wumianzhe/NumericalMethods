\documentclass[a4paper]{article}
\usepackage[T1,T2A]{fontenc}
\usepackage[russian]{babel}
\usepackage{amsmath, amsfonts, amssymb}
\usepackage[lmargin=5mm, rmargin=5mm]{geometry}
\usepackage{emptypage}

% epsilon is varepsion
\let\epsilon\varepsilon

\begin{document}
\begin{titlepage}
  \Large
  \begin{center}
    Санкт-Петербургский\\
    Политехнический университет Петра Великого\\
    \vspace{10em}
    Отчёт по лабораторной работе №1\\
    \vspace{2em}
    \textbf{Решение алгебраических и трансцендентных уравнений}
  \end{center}
  \vspace{6em}
  \begin{flushright}
    Студент: Копнов Александр Александрович\\
    Преподаватель: Добрецова Светлана Борисовна\\
    Группа: 5030102/00003
  \end{flushright}
  \vspace{\fill}
  \begin{center}
    Санкт-Петербург\\
    2021
  \end{center}
\end{titlepage}
\tableofcontents
\pagebreak

\section{Формулировка задачи}
Решить уравнения --- полиноминальное и трансцендентное методами половинного деления и секущих. Исследовать зависимость числа итераций и значения функции в найденной точке от заданой точности.
\section{Алгоритмы}
\subsection{Метод половинного деления}
Метод половинного деления подходит только для поиска корней кратности 1. Если на выбранном отрезке больше одного корня, найден будет только один.
\begin{enumerate}
  \item Задать концы отрезка $a,b$, функцию $f$, малое число $\epsilon >0$ (допустимая абсолютная погрешность), малое число $\delta>0$ (допуск связанный с точностью вычисления значения функции)\\
        Вычислить (или ввести) $f(a)$
  \item Вычислить $c = \frac{a+b}{2}$
  \item В цикле пока $|b-a|>2\epsilon$ и $|f(c)|>\delta$\\
        Если $f(a)\cdot f(c) < 0$, положить $b = c$, иначе положить $a = c$, $f(a) = f(c)$.
  \item После выхода из цикла считаем значение $c$ найденным корнем
\end{enumerate}
\subsection{Метод секущих}
\begin{enumerate}
  \item Задать $x_{0},x_{1}$, функцию $f$, малое число $\epsilon >0$ (допустимая погрешность)\\
        Если $f''(a) \cdot f(a) >0$ то можно принять $x_{0} = a, x_{1} = a+\delta$, иначе $x_{0}=b, x_{1} = b-\delta$\\
        $n=0$
  \item В цикле пока не выполнен критерий окончания\\
        $n=n+1$
        \[
        x_{n+1} = x_{n} + \frac{x_{n}-x_{n-1}}{f(x_{n-1})/f(x_{n})-1}
        \]
        Как критерий окончания можно использовать общий критерий ($|x_{n+1}-x_{n}| > \epsilon|x_{n+1}|$), который проще но может привести к лишним итерациям, или $|x_{n+1}-x_{n}| \leq \frac{M_{2}}{2m_{1}}|x_{n}-x_{n-1}|^{\phi}$
        где $M_{2} = \underset{x \in [a,b]}{\max{|f''(x)|}} < \infty, m_{1} = \underset{x \in [a,b]}{\min{|f'(x)|}}>0, \phi$~---золотое сечение ($\approx 1.61$)
  \item последний полученный $x$ считаем найденным корнем
\end{enumerate}
\section{Предварительный анализ задачи}
Даны уравнения: \[
  f(x) = 2x^{4} + 8x^{3} + 8x^{2} -1=0 \hspace{3cm} \phi(x) = (x-3)\cos(x)-1 =0
\]
Для полиноминального ур-я подберём промежуток поиска корней по теореме о верхней границе. Для положительных корней \[
  a' = -1, m = 4, a_{0} = 2 \Rightarrow x^{*} \leq 1 + \sqrt[4]{ \frac{1}{2} } \leq 1.840
\]
Нижняя граница положительных: $x = \frac{1}{y}$, $f(y) = 2y^{-4}+8^{-3}+8y^{-2}-1 \Rightarrow y^{4}-8y^{2}-8y-2 =0$
\[
  a' = -8, m=1, a_{0} = 1 \Rightarrow y^{*} \leq 1 + 8 \leq 9 \Rightarrow x \geq \frac{1}{9}
\]
Для отрицательных корней:\\
Нижняя граница: \[
  a' = -8, m=1, a_{0}=2 \Rightarrow x^{*} \geq -1 - 4 \Rightarrow x^{*} \geq -5
\]
Верхняя граница: \[
  a' = -8, m=1, a_{0}=1 \Rightarrow \frac{1}{x^{*}} \geq -1 -8 \Rightarrow x^{*} \leq \frac{1}{9}
\]
Найденный отрезок с отрицательными корнями содержит больше 1 корня, поэтому его рассматривать не будем и ограничимся отрезком $[0.111,1.840]$ на котором 1 корень

Для трансцендентного уравнения найдём промежутки поиска корней графически.\\
Функция имеет бесконечно много корней, но ограничимся одним, на промежутке от 5 до 5.4\\
\section{Проверка условий}
Метод половинного деления не имеет требований помимо наличия корня на промежутке, следовательно он применим к каждому промежутку.\\
Метод секущих:\\
\begin{enumerate}
  \item $f'(x),f''(x) \in C$; $\phi'(x),\phi''(x) \in C$
  \item a) $f''(\frac{1}{9})\cdot f(\frac{1}{9}) \approx -19.251 <0$\\
        b) $f''(1.840)\cdot f(1.840) \approx 1.83\cdot10^{4} > 0$\\
        c) $\phi''(5) \cdot \phi(5) \approx -0.58 < 0$\\
        d) $\phi''(5.4) \cdot \phi(5.4) \approx 0.017 > 0$
  \item $f''(x)$ и $\phi''(x)$ знакопостоянны на каждом промежутке
\end{enumerate}
Следовательно, метод секущих применим для поиска корней обоих уравнений.
\section{Тестовый пример}
Проведём поиск корня уравнения $x^{2}+3x+2=0$ на промежутке $x \in [-1.6;0]$ методом секущих\\
$f(-1,6)\cdot f(0) <0$, $f'(x)$ знакопостоянна на промежутке, следовательно, есть единственный корень.
$f''(x)=2 > 0$, значит метод секущих применим. Возьмём допустимую ошибку $\epsilon = 0.01$\\
$x_{0} =0, x_{1} = -0.01$\\
\[
  x_{2} = x_{1} + \frac{x_{1}-x_{0}}{f(x_{0})/f(x_{1}) -1} = -0.01 + \frac{f(-0.01)(-0.01)}{f(0)-f(-0.01)} = -0.6689
\]\[
  x_{3} = x_{2} + \frac{f(x_{2})(x_{2}-x_{1})}{f(x_{1})-f(x_{2})} = -0.8588
\]\[
  x_{4} = x_{3} + \frac{f(x_{3})(x_{3}-x_{2})}{f(x_{2})-f(x_{3})} = -0.9682
\]\[
  x_{5} = x_{4} + \frac{f(x_{4})(x_{4}-x_{3})}{f(x_{3})-f(x_{4})} = -0.9662
\]\[
  x_{6} = x_{5} + \frac{f(x_{5})(x_{5}-x_{4})}{f(x_{4})-f(x_{5})} = -0.9999 \approx 1
\]
\section{Подготовка контрольных тестов}
Были найдены корни каждого уравнения при значениях допустимой ошибки от $\epsilon = 10^{-1}$ до $\epsilon = 10^{-14}$.\\
При использовании метода половинного деления были заданы $a = 0.111, b=1.84$ для полинома, $a = 5, b = 5.4$ для трансцендентного ур-я.\\
При использовании метода секущих были заданы $x_{0}=1.84, x_{1}=1.8$ для полинома, $x_{0}=5.4, x_{1}=5.35$ для трансцендентного ур-я.
\section{Модульная структура программы}
poly$(x)$ --- функция, принимающая число $x$ и возвращающая значение полинома в данной точке\\
transc$(x)$ --- принимает число $x$ и возвращает значение трансцендентной функции в данной точке\\
secant$(f, x_0, x_1, eps)$ --- принимает функцию, корень которой ищем, первые 2 значения x и допустимую погрешность, возвращает найденный корень и кол-во итераций\\
bisection$(f, a, b, eps)$ --- принимает функцию, корень которой ищем, изначальные границы отрезка и допустимую погрешность, возвращает найденный корень и кол-во итераций\\
\section{Анализ результатов}
Из используемых методов (метод секущих, метод половинного деления, fzero) метод половинного деления сходится медленнее всего, на графиках логарифм значения функции и кол-во итераций зависят от логарифма допустимой погрешности
линейно (линейная сходимость).\\
Метод секущих сходится быстрее чем линейно, есть некоторое количество начальных итераций чтобы ``дойти'' до корня.\\
fzero требует примерно столько же итераций сколько метод половинного деления при малых точностях, но при уменьшении допустимой погрешности количество итераций растёт с примерно такой же скоростью как у метода секущих.
\section{Вывод}
Метод секущих и метод, используемый функцией fzero сходятся с практически одинаковой скоростью, но сложнее в реализации чем метод половинного деления.\\
Если удачно подобран промежуток, метод половинного деления может завершить работу почти сразу, попав в корень раньше чем будет достигнут предел для длины промежутка.
Метод секущих хорошо работает если начальные точки близко к корню. Также метод сходится тем быстрее чем ближе участок с корнем к прямой. На участках с большим количеством перепадов возможно ``перескочить'' через корень.
\end{document}
