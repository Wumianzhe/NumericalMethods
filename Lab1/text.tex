\documentclass[a4paper]{article}
\usepackage[T1,T2A]{fontenc}
\usepackage[russian]{babel}
\usepackage{amsmath, amsfonts, amssymb}
\usepackage[lmargin=5mm, rmargin=5mm]{geometry}
\usepackage{emptypage}

\begin{document}
\begin{titlepage}
  \Large
  \begin{center}
    Санкт-Петербургский\\
    Политехнический университет Петра Великого\\
    \vspace{10em}
    Отчёт по лабораторной работе №1\\
    \vspace{2em}
    \textbf{Приближение табличных функций}
  \end{center}
  \vspace{6em}
  \begin{flushright}
    Студент: Копнов Александр Александрович\\
    Преподаватель: Добрецова Светлана Борисовна\\
    Группа: 5030102/00003
  \end{flushright}
  \vspace{\fill}
  \begin{center}
    Санкт-Петербург\\
    2022
  \end{center}
\end{titlepage}
\tableofcontents
\pagebreak

\section{Формулировка задачи}
Задача: построить интерполяционный полином для гладкой функции $f(x) = \ctg x + x^{2}$ и функции с разрывом производной $g(x) = f(1.7) + |f(x) - f(1.7)|$ используя равномерную сетку. Исследовать влияние числа узлов и гладкости функции на качество интерполяции
\subsection{Формализация:}
Дана сетка $x^{h} := \{x_{i}\}_{i=0}^{n}$ и сеточная функция $y^{h}:=\{y_{i}\}_{i=0}^{n}$
Построить функцию $\varphi(x)$, удовлетворяющую критерию близости \[
  \varphi(x_{i}) = y_{i}, i=0,\ldots,n
\]
\section{Алгоритмы}
\textbf{Интерполяционный полином в форме Ньютона}
\[
  P_{n}(x) = y_{0} + (x-x_{0})y(x_{0},x_{1}) + \cdots + (x-x_{0})(x-x_{1})\cdots(x-x_{n-1})y(x_{0},\ldots,x_{n}) = \sum_{i=0}^n y(x_{0},x_{1},\ldots,x_{i}) \prod_{k=0}^{i-1} (x-x_{k})
\]
Где $y(x_{0},\ldots,x_{i})$ --- разделённая разность $i$-го порядка. \[
  y(x_{k_{0}},\ldots,x_{k_{m}}) = \frac{y(x_{k_{1}},\ldots,x_{k_{m}}) - y(x_{k_{0}},\ldots,x_{k_{m-1}})}{x_{k_{m}}-x_{k_{0}}}
\] В частности \[
  y(x_{0},\ldots,x_{i}) = \frac{y(x_{1},\ldots,x_{i}) - y(x_{0},\ldots,x_{i-1})}{x_{i}-x_{0}}
\]
\textbf{Условия применимости}
Интерполяционный процесс приводит к единственному решению если:
\begin{itemize}
  \item степень полинома на единицу меньше, чем кол-во точек
  \item $x_{i}$ попарно различны
\end{itemize}
\section{Предварительный анализ задачи}
По теореме Вейерштрасса любую непрерывную функцию можно приблизить полиномом с вещественными коэфициентами. Заданные
функции неперывны как сумма непрерывных
\section{Проверка условий}
Функции \(f\) и \(g\) непрерывны.
Полином Ньютона имеет степень, удовлетворяющую условиям по построению. Равномерная сетка также удовлетворяет условию попарного различия по построению
\section{Тестовый пример}
Произведём приближение функции $y = \ctg(x)+x^{2}$ на отрезке $[1;2.5]$ на сетке из 4 точек:\[
\begin{matrix}
x_{0} = 1& x_{1} = 1.5& x_{2} = 2& x_{3} = 2.5\\
y_{0} = 1.642& y_{1} = 2.321& y_{2} = 3.542& y_{3} = 4.911\\
\end{matrix}
\]
Разделённые разности: \[
  \begin{matrix}
    1.642 & 1.357 & 1.085 & -0.526\\
    2.321 & 2.443 & 0.295\\
    3.542 & 2.738\\
    4.911\\
  \end{matrix}
\]\[
  \frac{2.321 - 1.642}{1.5 - 1} = 1.357 \quad \frac{3.542 - 2.321}{2 - 1.5} = 2.443 \quad \frac{4.911 - 3.542}{2.5 - 2} = 2.738
\]\[
  \frac{2.443 - 1.357}{2 - 1} = 1.085 \quad \frac{2.738 - 2.443}{2.5 - 1.5} = 0.295
\]\[
  \frac{0.295 - 1.085}{2.5 - 1} = - 0.526
\]\[
  P_{n} = 1.642 + (x - 1)\cdot1.357 + (x-1)(x-1.5)\cdot1.085 + (x-1)(x-1.5)(x-2)\cdot(-0.526)
\]\[
  P_{n} = -0.526x^{3} + 3.452x^{2} - 4.774x + 3.49
\]
\section{Подготовка контрольных тестов}
Были построены приближения функций $f(x) = \ctg x + x^{2}$ и $g(x) = f(1.7) + |f(x) - f(1.7)|$.\\
Приближение строилось на отрезке $[1;2.5]$. Генерировались сетки заданного размера и значение между узлами сравнивалось со значением исходной функции.\\
По полученным результатам построены графики приближений полиномами 3,5 и 7 степеней, графики ошибки для этих же полиномов и график максимальной ошибки на промежутке в зависимости от степени полинома.
\section{Модульная структура программы}
$meshGen(a,b,n,f)$ --- генерация равномерной сетки на отрезке $[a,b]$ в которой $n+1$ узел и значения сеточной функции равны значениям функциии $f$. Возвращает сгенерированную сетку\\
$divDiff(mesh)$ --- подсчёт разделённых разностей на данной сетке $mesh$. Возвращает только используемые разности.\\
$newtonApprox(mesh, diff, x)$ --- по сетке $mesh$ и разностям $diff$ возвращает значение полинома в точке $x$.
\section{Анализ результатов}
Для гладкой функции ошибка падает, затем начинает возрастать когда степень полинома становится больше $\sim 25$. Наибольшая достигаемая точность имеет порядок $10^{-11}$\\
Для функции с разрывом первой производной ошибка уменьшается до $\sim 5$ степени полинома, затем начинает возрастать. Наибольшая достигаемая точность --- $\sim 10^{-1}$
\section{Вывод}
На равномерной сетке погрешность приближения непрерывной функции падает, затем начинает возрастать из-за нестабильности коэфициентов, возникающей из-за того, что каждое последующее слагаемое в форме Ньютона это корневой полином всё большей степени.

Погрешность приближения функции с разрывом производной изначально уменьшается за счёт уменьшения промежутков между узлами сетки, затем начинает играть всё большую роль возмущение от разрыва. Полином это гладкая функция и разрыв
в производной ведёт к тому, что нестабильность играет большую роль, так как необходимо компенсировать разницу в производной при малом изменении значения.

На графике заметны колебания точности с периодом около 15. Они связаны с тем, что узел сетки оказывается близок к точке разрыва, причём локальный максимум погрешности достигается когда узел совпадает с точкой разрыва.
Локальный минимум погрешности наблюдается когда точка разрыва находится примерно в середине отрезка.
\end{document}
