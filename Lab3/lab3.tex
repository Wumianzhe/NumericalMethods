\documentclass[a4paper]{article}
\usepackage[T1,T2A]{fontenc}
\usepackage[russian]{babel}
\usepackage{amsmath, amsfonts, amssymb}
\usepackage[lmargin=5mm, rmargin=5mm]{geometry}
\usepackage{emptypage}
\usepackage[varbb]{newpxmath}
\usepackage{mathtools}

\newcommand{\dd}[1][]{\ensuremath{\,\mathrm{d}^{#1}}}
\DeclarePairedDelimiter{\abs}{\lvert}{\rvert}

\begin{document}
\begin{titlepage}
  \Large
  \begin{center}
    Санкт-Петербургский\\
    Политехнический университет Петра Великого\\
    \vspace{10em}
    Отчёт по лабораторной работе №3\\
    \vspace{2em}
    \textbf{Решение интегралов с помощью квадратурных формул Ньютона-Котеса}
  \end{center}
  \vspace{6em}
  \begin{flushright}
    Студент: Копнов Александр Александрович\\
    Преподаватель: Добрецова Светлана Борисовна\\
    Группа: 5030102/00003
  \end{flushright}
  \vspace{\fill}
  \begin{center}
    Санкт-Петербург\\
    2022
  \end{center}
\end{titlepage}
\pagebreak

\section{Формулировка задачи}\label{sec:S1}
Задача: найти приближённое значение интеграла с заданной точностью с помощью обобщённой формулы трапеций для функции
\(f(x) = x^{5} - 2.2x^3 + 0.5x^2 -7x - 3.4\) на отрезке \([-2.0,-0.3]\). Исследовать зависимость фактической ошибки и
числа итераций от задаваемой по правилу Рунге точности. Исследовать зависимость фактической ошибки от длины отрезка разбиения.
\subsection{Формализация:}\label{subsec:SS1}

Представим определённый интеграл на промежутке \([a,b]\) функции \(F(x)\) в виде \[
  \int_a^b F(x)\dd x = \int_a^b p(x)f(x)\dd x
\]
где \(p(x)\) --- весовая функция. Необходимо вычислить приближённое значение интеграла, используя квадратурные формулы: \[
  \int_a^b p(x)f(x)\dd x \approx \sum_{k=1}^n A_{k}f(x_{k})
\]\(A_{k}\) и \(x_{k} \in [a,b]\) --- коэфициенты и узлы квадратурной формулы.

Если функция \(F\) имеет интегрируемые особенности, то за счёт \(p(x)\) их можно выделить.

\section{Алгоритм}\label{sec:S2}
\textbf{Формула трапеций}. Относится к квадратурным формулам Ньютона-Котеса, что означает выполнение двух
дополнительных условий:
\begin{enumerate}
  \item \(p(x) \equiv 1\)
  \item \(x_{k} = a + h(k-1), h = \frac{b-a}{n-1}\)
\end{enumerate}
Для одного интервала \(x_{1} = a, x_{2} = b, h = b-a\) \[
  \int_a^b f(x)dx \approx \frac{b-a}{2}(f(a) + f(b)) = h\left( \frac{1}{2}f(a) + \frac{1}{2}f(b) \right)
\]
Алгебраический порядок точности равен 1.\\
\textbf{Обобщённая формула трапеций}\\
Отрезок \([a,b]\) разбивается на \(N\) интервалов длиной \(h = \frac{b-a}{N}\). \[
  x_{k} = a + kh, k = \overline{0,N}
\]\[
  \int_a^b f(x)\dd x = \sum_{k=1}^N \int_{x_{k-1}}^{x_{k}}f(x)\dd x \approx \sum_{k=1}^N \frac{h}{2}(f(x_{k-1})+f(x_{k}))
\]
Обобщённая формула трапеций: \begin{equation} \label{eq:1}
  S_{2,N}(f) = \frac{h}{2}\left( f(a) + f(b) + 2 \sum_{k=1}^{N-1} f(x_{k}) \right) \tag{*}
\end{equation}
\textbf{Правило Рунге для оценки погрешности интеграла}\\
Остаточный член обобщённой квадратурной формулы Ньютона-Котеса имеет структуру \[
  R_{n,N}(f) = \alpha_{m}h^{m}f^{m_{1}}(\eta) = c_{m}h^{m}
\]
Пусть \(N_{1},N_2\) --- два разбиения отрезка \([a,b]\). На практике \(N_2 = 2N_1\). Если \(N_{2} > N_{1} \gg 1\),
\(f^{m_{1}}(\eta)\) --- ``среднее'' значение производной \(\implies c_{m}^{1} = c_{m}^2 = c_{m}\) \[
  \frac{h_{1}}{h_{2}} = \frac{N_{2}}{N_{1}} \implies c_{m}h^{m}_{2} = \frac{S_{n,N_{2}}(f)-S_{n,N_1}(f)}{\left(\frac{N_2}{N_1}\right)^{m} - 1}
\]
Условие достижения требуемой точности \(\epsilon\) \[
  \frac{\abs{S_{n,2N}(f)-S_{n,N}(f)}}{2^{m} - 1} < \epsilon
\]
Для формулы трапеций \(m = 2\).
\section{Предварительный анализ задачи}\label{sec:S3}

Для того, чтобы функция была интегируемой, она должна быть ограничена, \(f(x)\) ограничена на \([a,b]\), для оценки
точности используется вторая производная.
\section{Тестовый пример}\label{sec:S5}

Найдём интеграл для функции \(f(x) = x^{5} -2.2x^3 + 0.5x^2 -7x -3.4\) на отрезке \([-2.0, -0.3]\) используя обобщённую
формулу парабол.
\begin{enumerate}
  \item 1 отрезок разбиения, тогда узлы: \(x_{0} = -2.0, x_{1} = -0.3, h = \frac{-0.3 + 2.0}{1} = 1.7\)\\
        Тогда согласно формуле~\eqref{eq:1} \[
        \int_{-2.0}^{-0.3}f(x)\dd x \approx \frac{1.7}{2} (-1.800 + -1.198) = -2.548
        \]
  \item 2 отрезка разбиения, узлы: \(x_0 = -2.0, x_1 = -1.15, x_2 = -0.3, h = \frac{-0.3+2.0}{2} = 0.85\)\\
        Тогда согласно формуле~\eqref{eq:1} \[
        \int_{-2.0}^{-0.3} f(x)\dd x \approx \frac{0.85}{2}(-1.800 + -1.198 + 2 \cdot 6.646) = 4.375
        \]
        По правилу Рунге точность равна \(\frac{4.375 + 2.548}{3} = 2.298\)
  \item 4 отрезка разбиения, узлы:
        \(x_0 = -2.0, x_1 = -1.575, x_2 = -1.15, x_3 = -0.725, x_4 = -0.3, h = \frac{-0.3+2.0}{4} = 0.425\)\\
        Тогда согласно формуле~\eqref{eq:1} \[
        \int_{-2.0}^{-0.3} f(x)\dd x \approx \frac{0.425}{2}(-1.800 + -1.198 + 2 \cdot (7.769 + 6.656 + 2.576)) = 6.584
        \]
        По правилу Рунге точность равна \(\frac{6.584 - 4.375}{3} = 0.736\).
\end{enumerate}
Вычислив этот интеграл по формуле Ньютона-Лейбница получим \[
  \int_{-2.0}^{-0.3} f(x)\dd x = \left( \frac{x^{6}}{6} - \frac{11x^{4}}{20} + \frac{x^3}{6} - \frac{7x^{2}}{2} - \frac{17}{5} x \right)\bigg|_{-2.0}^{-0.3} = 7.363
\]
Расхождение интеграла и точности по Рунге на первом переходе можно объяснить тем, что правило Рунге предполагает большое
кол-во отрезков, ведущее к усреднению производной и совпадению констант. На переходе от двух отрезков к четырём
полученная по формуле Рунге точность уже близка к фактической ошибке.
\section{Подготовка контрольных тестов}\label{sec:S6}
Проводилось численное интергирование функции \(f\) на отрезке \([-2.0,-0.3]\). Значение интеграла находилось с заданной длиной шага.\\
Достижение заданной точности, изменяемой от \(10^{1}\) до \(10^{-13}\), проверялось по ф-ле Рунге сравнением значений с разным шагом.
\section{Модульная структура программы}\label{sec:S7}
\(integrate(a,b,f,k)\) --- возвращает приближённое значение интеграла полученной функции \(f\) от \(a\) до \(b\) с шагом \(h=\frac{b-a}{2^{k-1}}\)
\section{Анализ результатов}\label{sec:S8}
Заданная точность достигается вплоть до \(\epsilon = 10^{-11}\), после этого фактическая ошибка остаётся на значении \(1.97\cdot10^{-12}\).
Ошибка меняется пропорционально квадрату длины шага, что соответствует теоритическим предсказаниям.
Для достижения точности на три порядка больше требовалось 5 дополнительных итераций.
\section{Вывод}\label{sec:S9}

Практические результаты совпадают с теоретическими предсказаниями, точность достигается до машинной. Количество вычислений растёт пропорционально корню требуемой точности
(кол-во вычислений выросло в \(2^{5} = 32\) раза, точность вырасла в \(1000 \approx 1024 = 32^{2}\) раз)
\end{document}
