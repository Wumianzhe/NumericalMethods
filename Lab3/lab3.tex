\documentclass[11pt,a4paper]{article}
\usepackage[T1,T2A]{fontenc}
\usepackage[russian]{babel}
\usepackage{amsmath, amsfonts, amssymb}
\usepackage[lmargin=5mm, rmargin=5mm]{geometry}
\usepackage{emptypage}
\usepackage{booktabs}
\usepackage{graphicx}

% epsilon is varepsion
\let\epsilon\varepsilon
\renewcommand\;{\hspace{1cm}}

\begin{document}
\begin{titlepage}
  \Large
  \begin{center}
    Санкт-Петербургский\\
    Политехнический университет Петра Великого\\
    \vspace{10em}
    Отчёт по лабораторной работе №3\\
    \vspace{2em}
    \textbf{Решение Систем Линейных Алгебраических Уравнений\\
    итерационными методами}
  \end{center}
  \vspace{6em}
  \begin{flushright}
    Студент: Копнов Александр Александрович\\
    Преподаватель: Добрецова Светлана Борисовна\\
    Группа: 5030102/00003
  \end{flushright}
  \vspace{\fill}
  \begin{center}
    Санкт-Петербург\\
    2021
  \end{center}
\end{titlepage}
\tableofcontents
\pagebreak

\section{Формулировка задачи и её формализация}
\textbf{Задача:} найти решение $X$ системы линейных алгебраических уравнений $AX = B$ с точностью $\epsilon$, т.е. $||X^{*}-X|| < \epsilon$ методом простых итераций с оптимальным параметром. Исследовать зависимость числа итераций от заданной точности.
\textbf{Формализация:}\\
Дана СЛАУ $AX = B$, где $A \in M_{n \times n}$, $B \in M_{n \times 1}$\\
Если $det(A) \neq 0 \Rightarrow \exists! x^{*} \in M_{n\times1} : Ax^{*} = B$\\
Найти $x : ||x^{*}-x|| < \epsilon$
\section{Алгоритм}
Пусть имеется СЛАУ: \[
  \begin{cases}
    a_{11}x_{1} + a_{12}x_{2} + \cdots + a_{1n}x_{n} = b_{1}\\
    a_{21}x_{1} + a_{12}x_{2} + \cdots + a_{2n}x_{n} = b_{2}\\
    \cdots \cdots \cdots \cdots \cdots \cdots \cdots \cdots \cdots \cdots \cdots  \\
    a_{n1}x_{1} + a_{n2}x_{2} + \cdots + a_{nn}x_{n} = b_{n}\\
  \end{cases}
\]
Итерационные методы можно записать при помощи формулы \[
  x^{(k+1)}=x^{(k)} - \alpha_{k}B_{k}^{-1}(Ax^{k}-b)=(E - \alpha_{k}B^{-1}_{k}A)x^{(k)} + \alpha_{k}B^{-1}_{k}b = C_{k}x^{(k)} + g_{k}
\]
В методе простых итераций используются постоянные параметры $\alpha = \frac{2}{\lambda_{1} + \lambda_{n}}$, где $\lambda_{1}$ --- минимальное собственное число, $\lambda_{n}$ --- максимальное. $B = E$, соответственно формула принимает вид \[
  x^{(k+1)} = (E - \alpha A)x^{(k)} + \alpha b = Cx^{(k)} + g
\]
Необходимым условием сходимости является $ \forall |\lambda_{C}| < 1$.
Условие останова итераций для достижения заданной точности: \[
  \frac{||C||}{1-||C||}||x^{(k+1)}-x^{(k)}|| < \epsilon
\]
В этом неравенстве используется та норма, в которой выполняется достаточное условие сходимости: $||C|| < 1$.
\section{Предварительный анализ задачи}
Матрица коэфициентов СЛАУ не является вырожденной, т.е. $\det{A} \neq 0$. Матрица $A$ строится как $A = Q^{T}DQ$,
где матрица $D$ --- диагональная не вырожденная, $Q$ --- ортогональная.
\section{Проверка условий}
Метод простых итераций требует, чтобы матрица была положительно определённой и симметричной.\\
Условия положительной определённости и симметричности выполняются по построению.
\section{Тестовый пример с рассчётами}
Решим СЛАУ Методом простых итераций: \[
  A = \begin{pmatrix}
    3 & -1 & 1\\
    -1 & 5 & 1\\
    1 & 1 & 4
  \end{pmatrix} \quad X^{*} = \begin{pmatrix}
    1 \\ 2 \\ -1
  \end{pmatrix} \quad b = \begin{pmatrix}
    0 \\ 8 \\ -1
  \end{pmatrix}
\]
\[
  \alpha = \frac{2}{\lambda_{1} + \lambda_{n}} = \frac{2}{1.79 + 5.68} = 0.2681
\]\[
  C = E - \alpha A = \begin{pmatrix}
    0.196 & 0.268 & -0.268\\
    0.268 & -0.340 & -0.268\\
    -0.268 & -0.268 & -0.072\\
  \end{pmatrix}
\]
Так как $||C||_{\infty} = 0.8765 < 1$, можно использовать бесконечную норму для меньшей погрешности вычислений.
За начальное приближение возьмём $g = \alpha b$
\[
  x_{0} = \begin{pmatrix}
    0 \\ 2.14 \\ -0.268
  \end{pmatrix} \; \frac{||C||_{\infty}}{1-||C||_{\infty}}||x^{k+1}-x^{k}||_{\infty} = 7.10 ||x^{k+1}-x^{k}||_{\infty} < \epsilon
\]\[
  x_{1} = Cx_{0} + g = \begin{pmatrix}
    0.196 & 0.268 & -0.268\\
    0.268 & -0.340 & -0.268\\
    -0.268 & -0.268 & -0.072\\
  \end{pmatrix} \begin{pmatrix}
    0 \\ 2.14 \\ -0.268
  \end{pmatrix} + \begin{pmatrix}
    0 \\ 2.14 \\ -0.268
  \end{pmatrix} = \begin{pmatrix}
      0.65 \\ 1.49 \\ -0.82
    \end{pmatrix}
\]\[
  x_{2} = Cx_{1} + g = \begin{pmatrix}
    0.196 & 0.268 & -0.268\\
    0.268 & -0.340 & -0.268\\
    -0.268 & -0.268 & -0.072\\
  \end{pmatrix} \begin{pmatrix}
      0.65 \\ 1.49 \\ -0.82
    \end{pmatrix} + \begin{pmatrix}
    0 \\ 2.14 \\ -0.268
  \end{pmatrix} = \begin{pmatrix}
      0.7459 \\ 2.03 \\ -0.78
    \end{pmatrix}
  \]\[
  x_{3} = Cx_{2} + g = \begin{pmatrix}
    0.196 & 0.268 & -0.268\\
    0.268 & -0.340 & -0.268\\
    -0.268 & -0.268 & -0.072\\
  \end{pmatrix} \begin{pmatrix}
      0.7459 \\ 2.03 \\ -0.78
    \end{pmatrix} + \begin{pmatrix}
    0 \\ 2.14 \\ -0.268
  \end{pmatrix} = \begin{pmatrix}
      0.9 \\ 1.86 \\ -0.95
    \end{pmatrix}
\]
\begin{center}
  \begin{tabular}{c | c | c | c | c | c | c | c}
    № итерации & 0 & 1 & 2 & 3 & 4 & 5 & 6\\
    \hline
    $x$ & $\begin{pmatrix}
      0 \\ 2.14 \\ -0.268
    \end{pmatrix}$ & $\begin{pmatrix}
      0.65 \\ 1.49 \\ -0.82
    \end{pmatrix}$ & $\begin{pmatrix}
      0.7459 \\ 2.03 \\ -0.78
    \end{pmatrix}$ & $\begin{pmatrix}
      0.9 \\ 1.86 \\ -0.95
    \end{pmatrix}$ & $\begin{pmatrix}
      0.93 \\ 2.01 \\ -0.94
    \end{pmatrix}$ & $\begin{pmatrix}
      0.97 \\ 1.96 \\ -0.99
    \end{pmatrix}$ & $\begin{pmatrix}
      0.98 \\ 2.0 \\ -0.98
    \end{pmatrix}$\\
    \hline
    $\epsilon$ & & 4.67 & 3.87 & 1.25 & 1.04 & 0.35 & 0.28
  \end{tabular}
\end{center}
\section{Подготовка контрольных тестов}
Было случайно сгенерировано точное решение $x^{*}$, $x^{*}_{i} \in [-10,10]$ затем сгенерированы матрицы коэфициентов удовлетворяющие условиям применимости. Левая часть системы получена как $b = Ax^{*}$. Каждая СЛАУ была решена МПИ.
По полученным результатам построены графики зависимости итераций от $\epsilon$ и зависимости нормы ошибки $||x-x^{*}||_{2}$ и невязки $||Ax - b||_{2}$ от $\epsilon$ при $\epsilon$ от $10^{-1}$ до $10^{-13}$
\section{Модульная структура программы}
Определены арифметические операции с матрицами, умножение матрицы на число и умножение матриц.\\
$norm(v)$ --- вторая норма вектора\\
$infnorm(v)$ --- бесконечная норма вектора или матрицы\\
$optAlpha(A)$ --- оптимальный параметр для матрицы $A$\\
$matNorm2(A)$ --- вторая норма матрицы $A$\\
$optFpi(A,b,eps,x_{0})$ --- МПИ, принимает СЛАУ, требуемую точность и начальное приближение, возвращает столбец-решение\\
$posDefGen(size)$ --- генерация симметричной п.о. матрицы заданного размера
\section{Анализ результатов}
\begin{figure}[h]
\includegraphics[width = 0.5\textwidth]{precision.pdf}
\includegraphics[width = 0.5\textwidth]{iterations.pdf}
\end{figure}
Из графиков видно, что количество итераций линейно зависит от $\log{\epsilon}$, нормы ошибки и невязки близки к $\epsilon$, при этом ошибка примерно на порядок меньше чем невязка.

При заданной точности $\epsilon = 10^{-10}$, ошибка составила $1.53 \cdot 10^{-11}$, невязка --- $9.40 \cdot 10^{-11}$. Точность достигается.
\section{Вывод}
МПИ сходится довольно медленно, для уточнения решения на порядок требуется порядка 10-11 итераций.
На МПИ сильно влияет погрешность вычислений, и он может не достигать точности $10^{-14}$ или выше, конкретный результат зависит от сгенерированной матрицы. Это особенно видно при использовании второй нормы. По невязке можно хорошо оценить фактическую ошибку.

При использовании нормы, отличной от второй, погрешность вычислений меньше и точности до $10^{-15}$ включительно достигались. Однако, это ведёт к более строгим условиям для применимости.
\end{document}
