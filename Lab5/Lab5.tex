\documentclass[a4paper]{article}
\usepackage[T1,T2A]{fontenc}
\usepackage[russian]{babel}
\usepackage{amsmath, amsfonts, amssymb}
\usepackage[lmargin=5mm, rmargin=5mm]{geometry}
\usepackage{emptypage}

\usepackage[varbb]{newpxmath}
\usepackage{mathtools}

\newcommand{\dd}[1][]{\ensuremath{\,\mathrm{d}^{#1}}}
\DeclarePairedDelimiter{\abs}{\lvert}{\rvert}
\begin{document}
\begin{titlepage}
  \Large
  \begin{center}
    Санкт-Петербургский\\
    Политехнический университет Петра Великого\\
    \vspace{10em}
    Отчёт по лабораторной работе №5\\
    \vspace{2em}
    \textbf{Решение задачи Коши для ОДУ 1 порядка методами Рунге-Кутты}
  \end{center}
  \vspace{6em}
  \begin{flushright}
    Студент: Копнов Александр Александрович\\
    Преподаватель: Добрецова Светлана Борисовна\\
    Группа: 5030102/00003
  \end{flushright}
  \vspace{\fill}
  \begin{center}
    Санкт-Петербург\\
    2022
  \end{center}
\end{titlepage}
\pagebreak
\section{Формулировка задачи}\label{sec:S1}
Задача: найти численное решение задачи Коши на равномерной сетке с помощью метода Рунге-Кутты 4 порядка с коэфициентом
\(\frac{1}{2}\).

Исследовать зависимость фактической точности и числа итераций от заданной точности, зависимость нормы ошибки от величины
возмущения при заданной точности, зависимость фактической точности от величины шага. По графику определить порядок
точности метода и константу
\subsection{Формализация:}\label{subsec:SS1}
Задача Коши ставится для дифференциального уравнения первого порядка с начальным условием \[
  y' = f(x,y) \quad x \in [a,b] \quad y(a) = y_{a}
\]
\section{Алгоритм}\label{sec:S2}
\textbf{Метод 4 порядка с коэфициентом 1/2}.\\
Строится сетка на отрезке \([a,b]\), \(x_{k} = a + kh, k = \overline{0,n}, h = \frac{b-a}{n}\)

Метод строится по схеме \[
  y_{i+1} = y_{i} + \frac{h}{6}(k_1+2k_2+2k_3+k_4) \quad i = \overline{0,n-1} \quad y_{0} = y_{a}
  \text{ --- известно }
\]\[
  \begin{matrix}
    k_1 = f(x_{i},y_{i})\\
    k_2 = f(x_{i}+\frac{h}{2}, y_{i} + \frac{h k_{1}}{2})\\
    k_3 = f(x_{i}+\frac{h}{2}, y_{i} + \frac{h k_{2}}{2})\\
    k_4 = f(x_{i}+h, y_{i} + h k_3)
  \end{matrix}
\]
Величина шага может быть изменена при переходе от одной точки к другой. Один из способов вычислить значение функции с
определённой точностью --- применить правило Рунге. Тогда алгоритм действий для получения следующего значения будет таким:
\begin{enumerate}
  \item \(k = 1\)
        \(h = x_{i+1}-x_{i}\).\\
        Вычислить \(y_{p}\) по схеме выше с шагом \(h\) (\(y_{p} = step(x_{i},y_{i},h)\))\\
        Вычислить \(y_{n} = step(x_{i},y_{i},\frac{h}{2})\)\\
        \(y_{n} = step(x_{i}+\frac{h}{2},y_{n},\frac{h}{2})\) (так как требуется два шага)\\
  \item  Пока поправка \(\frac{|y_{p}-y_{n}|}{15} > \epsilon\)\\
        \(y_{p} = y_{n}\)\\
        \(k = k+1\)\\
        \(h = \frac{x_{i+1}-x_{i}}{2^{k}}\)\\
        \(y_{n} = y_{i}\)\\
        \(for\ j\ in\ \overline{0,2^{k}-1}\)\\
        \hspace*{0.5cm}\(y_{n} = step(x_{i} + jh,y_{n}, h)\)
        \item  Значение функции --- \(y_{n}\), количество итераций --- \(k\), можно переходить к следующей точке.
\end{enumerate}
\textbf{Условия применимости:}\\
Функция \(f(x,y)\) должна удовлетворять условию Липшица по \(y\) на \([a,b]\), т.е. \[
  \exists L > 0\ :\ |f(x,y_1) - f(x,y_2)| \leq L|y_1-y_2|
\] (единственность решения задачи Коши)
\section{Предварительный анализ задачи}\label{sec:S3}
Задача Коши\[
  y = x(y' - x\cos x) \quad y' = \frac{y}{x} + x\cos x \quad x \in [\frac{\pi}{2},2\pi] \quad y(\frac{\pi}{2}) = \frac{\pi}{2}
\]
Имеет единственное решение \(y = x\sin x\)
\section{Тестовый пример}\label{sec:S5}
Вычислим значения функции в точке \(\frac{\pi}{2}+0.2\) с шагом \(0.2\) и \(0.1\)\\
1. \(h = 0.2\)\[
\begin{matrix}
  k_1 = f(\frac{\pi}{2},\frac{\pi}{2}) = 1 + \frac{\pi}{2} \cdot 0 = 1\\
  k_2 = f(\frac{\pi}{2} + 0.1, \frac{\pi}{2} + 0.1) = 1 + (\frac{\pi}{2} + 0.1)\cdot(-0.100) = 0.833\\
  k_3 = f(\frac{\pi}{2} + 0.1, \frac{\pi}{2} + 0.083) = 1.010 -0.100 (\frac{\pi}{2} + 0.1) = 0.823\\
  k_4 = f(\frac{\pi}{2} + 0.2, \frac{\pi}{2} + 0.2\cdot0.823) = 1.018 -0.199(\frac{\pi}{2}+0.2) = 0.628
   \end{matrix}
 \]\[
   y(\frac{\pi}{2}+0.2) = \frac{\pi}{2} + \frac{0.2}{6}(1 + 2*0.833 + 2*0.843 + 0.628) = 1.7368
 \]
 2. \(h = 0.1\) \[
   \begin{matrix}
     k_1 = 1\\
     k_2 = f(\frac{\pi}{2} + 0.05, \frac{\pi}{2} + 0.05) = 0.918\\
     k_3 = f(\frac{\pi}{2} + 0.05, \frac{\pi}{2} + 0.046) = 0.916\\
     k_4 = f(\frac{\pi}{2} + 0.1,\frac{\pi}{2} + 0.1*0.916) = 0.828
   \end{matrix}
 \]\[
   y(\frac{\pi}{2} + 0.1) = \frac{\pi}{2} + \frac{0.1}{6}(1 + 2*0.918 + 2*0.916 + 0.828) = 1.662
 \]\[
   \begin{matrix}
     k_1 = f(1.670,1.662) = 0.830\\
     k_2 = f(1.720,1.703) = 0.734\\
     k_3 = f(1.720,1.698) = 0.731\\
     k_4 = f(1.770,1.735) = 0.629\\
   \end{matrix}
 \]\[
   y(\frac{\pi}{2} + 0.2) = 1.662 + \frac{0.1}{6}(0.830+2*0.734+2*0.731+0.629) = 1.73515
 \]
 Ошибка по рунге \(\delta = \frac{|1.7368 - 1.73515|}{15} = 1.1\cdot10^{-4}\)\\
 фактическая ошибка: \(y(\frac{\pi}{2}+0.2) = 1.7355\), \(\Delta \approx 3.5\cdot10^{-4}\)

 \section{Подготовка контрольных тестов}\label{sec:S6}
 Было найдено решение задачи Коши на отрезке \([\frac{\pi}{2},2\pi]\) с начальным шагом \(h = \frac{2\pi-\frac{\pi}{2}}{25} \approx 0.188\),
 уменьшаемым в два раза 10 раз.
 Затем на том же отрезке была найдена максимальная ошибка при вычислении с заданной точностью \(\epsilon = 10^{-k}\),
 \(k = \overline{1,15}\)\\
 Также было исследовано поведение ошибки при возмущении граничных условий \(y_{0}=y(a)\cdot(1-\Delta)\), где \(\Delta\) изменялась от
 \(10^{-15}\) до \(10^{-1}\) при заданной точности \(\epsilon = 10^{-9}\)

\section{Модульная структура программы}\label{sec:S7}
\(step(y_{p},x,h,f)\) --- вычисление \(y_{i+1}\) при \(y_{i} = y_{p}, x_{i} = x\), шаге \(h\), где \(y' = f(x,y)\)\\
\(epsStep(y_{p},x_{p},x_{n},eps,f)\) --- вычисление \(y_{i+1}\) при \(y_{i} = y_{p}, x_{i} = x_{p}, x_{i+1} = x_{n}\) с
заданной точностью \(eps\), где \(y' = f(x,y)\). Возвращает значение функции и число итераций\\
\section{Анализ результатов}\label{sec:S8}
Ошибка на отрезке минимальна на левой границе и возрастает по мере приближения к правой границе. При шаге
\(1.88*10^{-1}\) фактическая ошибка \(\Delta = 6.18*10^{-6}\) и уменьшается примерно в 16 раз при уменьшении шага в два раза.

При задании точности по правилу Рунге фактическая ошибка больше, чем заданная примерно на порядок, однако разница
колеблется.

При увеличении требуемой точности на порядок требуется одна дополнительная итерация, но есть один переход в котором
дополнительной итерации не потребовалось

При возмущении входных данных наименьшая ошибка достигается при возмущении \(\Delta \approx 6\cdot10^{-10}\). При меньшем возмущении
ошибка примерно постоянна, при большем она зависит от возмущения линейно.

При изменении шага величина ошибки параллельна прямой \(h^{4}\), вплоть до достижения машинной точности \(\Delta \approx 2\cdot10^{-14}\)
\section{Вывод}\label{sec:S9}
Уменьшение ошибки в 16 раз при уменьшении шага в 2 раза соответствует теории, \(\Delta \sim h^{4}= h^{s}\), где
\(s\)---порядок метода. Фактически ошибка примерно равна \(6\cdot10^{-3}\cdot h^{4}\). Коэфициент может зависеть от выбранного
отрезка, заданной функции \(f\)

Минимум ошибки на левой границе обусловлен наличием там точного решения \(y_{0} = y(a)\).

При задании точности по правилу Рунге ошибка больше из-за того, что ошибки на отрезке складываются, а правило Рунге
ограничивает локальную ошибку. Колебание отношения вызвано тем, что локальные ошибки могут иметь разные знаки, поэтому
фактическое поведение ошибки предсказать сложно.

При задании точности должно требоваться 5 итераций на 6 порядков (т.к. \(2^{4\cdot5} \approx 10^{6}\)), но так как ошибка с
начальным шагом достаточна вплоть до \(\epsilon = 10^{-7}\), до достижения машинной точности требуется только \(7\)
дополнительных итераций, соответственно нельзя сделать точный вывод по одному переходу без дополнительной итерации.

При возмущении входных данных на \(\Delta \approx 6\cdot10^{-10}\), что немного меньше заданной точности, ошибка меньше ошибки без
возмущения примерно на пол порядка и почти равна заданной точности.
\end{document}
