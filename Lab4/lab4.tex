\documentclass[11pt,a4paper]{article}
\usepackage[T1,T2A]{fontenc}
\usepackage[russian]{babel}
\usepackage{amsmath, amsfonts, amssymb}
\usepackage[lmargin=5mm, rmargin=5mm]{geometry}
\usepackage{emptypage}
\usepackage{booktabs}
\usepackage{graphicx}

% epsilon is varepsion
\let\epsilon\varepsilon
\renewcommand\;{\hspace{1cm}}

\begin{document}
\begin{titlepage}
  \Large
  \begin{center}
    Санкт-Петербургский\\
    Политехнический университет Петра Великого\\
    \vspace{10em}
    Отчёт по лабораторной работе №4\\
    \vspace{2em}
    \textbf{Решение Алгебраической проблемы собственных значений}
  \end{center}
  \vspace{6em}
  \begin{flushright}
    Студент: Копнов Александр Александрович\\
    Преподаватель: Добрецова Светлана Борисовна\\
    Группа: 5030102/00003
  \end{flushright}
  \vspace{\fill}
  \begin{center}
    Санкт-Петербург\\
    2021
  \end{center}
\end{titlepage}
\tableofcontents
\pagebreak

\section{Формулировка задачи и её формализация}
\textbf{Задача:} Дана квадратная матрица $A$. Нужно найти все или часть её собственных чисел (с.ч.) и соответствующие им собственные вектора (с.в.) методом Якоби с оптимальным выбором ключевого элемента с точностью $\epsilon$.
\textbf{Формализация:}\\
Дана матрица $A$, найти $\lambda, x : Ax = \lambda x$.
Для вычисленных $\lambda, x$ и точных $\lambda^{*},x^{*}$ должно выполняться условие $|\lambda - \lambda^{*}| < \epsilon, ||x - x^{*}|| < \epsilon$
\section{Алгоритм}
\subsection{Метод Якоби}
Для симметричной матрицы $A$ существует полная ортонормированная система векторов. Следовательно, матрица $X$ из таких векторов является ортогональной $(X^{-1} = X^{T})$, запишем как следствие равенство \[
  \Lambda = X^{-1}AX = X^{T}AX
\]
значит, для всякой матрицы $A$ найдётся ортогонально подобная ей матрица $\Lambda$ состоящая из собственных чисел матрицы $A$. Чтобы привести матрицу $A$ к матрице $\Lambda$ будем использовать ортогональные преобразования с помощью матриц плоских вращений $T_{ij}$.
Выполним последовательно умножение матриц (в силу симметричности матрицы $A$, $a_{ji} = a_{ij} \forall i,j$)\[
  AT_{ij} = \begin{pmatrix}
    a_{11} & \cdots & a_{1i} & \cdots & a_{1j} & \cdots & a_{1n}\\
    \vdots & & \vdots & & \vdots & & \vdots \\
    a_{1i} & \cdots & a_{ii} & \cdots & a_{ij} & \cdots & a_{in}\\
    \vdots & & \vdots & & \vdots & & \vdots \\
    a_{1j} & \cdots & a_{ij} & \cdots & a_{jj} & \cdots & a_{jn}\\
    \vdots & & \vdots & & \vdots & & \vdots \\
    a_{1n} & \cdots & a_{in} & \cdots & a_{jn} & \cdots & a_{nn}\\
  \end{pmatrix} \begin{pmatrix}
    1 & \cdots & 0 & \cdots & 0 & \cdots & 0\\
    \vdots & & \vdots & & \vdots & & \vdots \\
    0 & \cdots & c & \cdots & s & \cdots & 0\\
    \vdots & & \vdots & & \vdots & & \vdots \\
    0 & \cdots & -s & \cdots & c & \cdots & 0\\
    \vdots & & \vdots & & \vdots & & \vdots \\
    0 & \cdots & 0 & \cdots & 0 & \cdots & 1\\
\end{pmatrix} =
\]\[
  \begin{pmatrix}
    a_{11} & \cdots & ca_{1i} - sa_{1j} & \cdots & sa_{1i}+ ca_{1j}& \cdots & a_{1n}\\
    \vdots & & \vdots & & \vdots & & \vdots \\
    a_{1i} & \cdots & ca_{ii}  - sa_{ij}& \cdots & sa_{ii} + ca_{ij}& \cdots & a_{in}\\
    \vdots & & \vdots & & \vdots & & \vdots \\
    a_{1j} & \cdots & ca_{ij}  - sa_{jj}& \cdots & sa_{ij} + ca_{jj}& \cdots & a_{jn}\\
    \vdots & & \vdots & & \vdots & & \vdots \\
    a_{1n} & \cdots & ca_{in}  - sa_{jn}& \cdots & sa_{in} + ca_{jn}& \cdots & a_{nn}\\
  \end{pmatrix}
\]\[
  B = T_{ij}^{T}AT_{ij} =
  \begin{pmatrix}
    a_{11} & \cdots & ca_{1i} - sa_{1j} & \cdots & sa_{1i}+ ca_{1j}& \cdots & a_{1n}\\
    \vdots & & \vdots & & \vdots & & \vdots \\
    ca_{1i} - sa_{1j} & \cdots & c^{2}a_{ii}  + s^{2}a_{ij} - 2csa_{ij}& \cdots & cs(a_{ii}-a_{jj}) + (c^{2} - s^{2})a_{ij} & \cdots & ca_{in}  - sa_{jn}\\
    \vdots & & \vdots & & \vdots & & \vdots \\
    sa_{1i}+ ca_{1j} & \cdots & cs(a_{ii}-a_{jj}) + (c^{2} - s^{2})a_{ij}& \cdots & s^{2}a_{ii} + c^{2}a_{jj} + 2csa_{ij}& \cdots & sa_{in} + ca_{jn}\\
    \vdots & & \vdots & & \vdots & & \vdots \\
    a_{1n} & \cdots & ca_{in}  - sa_{jn}& \cdots & sa_{in} + ca_{jn}& \cdots & a_{nn}\\
  \end{pmatrix}
\]
Числа $c$ и $s$ подбираются такими, что $c^{2} + s^{2} = 1$ и в преобразованной матрице появился нуль на месте элемента $a_{ij}$. Если рассматривать $s,c$ как синус и косинус угла $\alpha$, то $\tg(2\alpha) = \frac{2a_{ij}}{a_{jj}-a_{ii}}$.
Однако, угол искать не требуется, и для уменьшения погрешности вычисления использовались следующие формулы:\[
  p = 2a_{ij} \; q = a_{jj}-a_{ii} \; d = \sqrt{p^{2}+q^{2}} \; r = \frac{|q|}{2d}
\]\[
  c = \sqrt{0.5+r} \; s = \sqrt{0.5-r}\cdot sign(pq)
\]
Элемент выбирается в некотором смысле оптимальным способом:\\
$a_{i_{0}j_{0}}$ --- оптимальный элемент, \[
  i_{0}: r_{i_{0}} = \underset{k}{max}\: r_{k} = \underset{k}{max} \sum_{j=0,j\neq k}^{n} a_{kj}^{2}
\]\[
  j_{0}: |a_{i_{0}j_{0}}| = \underset{j,j\neq i_{0}}{max} |a_{i_{0}j}|
\]

Операция обнуления элемента продолжается, пока $\sum_{i>j}a^{2}_{ij} > \epsilon$.
\subsection{Метод обратных итераций со сдвигами}
Метод обратных итераций со сдвигами применяют когда надо с большой точностью найти собственный вектор, соответствующий известному собственному числу
\section{Предварительный анализ задачи}
Матрица $A$ симметричная, строится как $A = Q^{T}DQ$, где $D$ --- диагональная матрица собственных значений (матрица $\Lambda$), $Q$ --- ортогональная, полученная как матрица Хаусхолдера из случайного нормированного вектора.
\section{Проверка условий}
Метод Якоби с оптимальным выбором ключевого элемента требует только симметричности матрицы, условие выполняется по построению.
\section{Тестовый пример с рассчётами}
Найдём методом Якоби с оптимальным выбором ключевого элемента собственные числа матрицы \[
  A = \begin{pmatrix}
    3.09106 & -0.413837 & -1.30054\\
    -0.413837 & 2.06374 & -1.15659\\
    -1.30054 & -1.15659 & 3.8452
  \end{pmatrix} \; \Lambda = \begin{pmatrix}
    1 \\ 3 \\ 5
  \end{pmatrix}
\]
Оптимальный элемент --- $a_{31} = -1.30054$, $c = 0.800, s = 0.6$ \[
  T_{31} = \begin{pmatrix}
    0.8 & 0 & 0.6 \\
    0 & 1 & 0\\
    -0.6 & 0 & 0.8\\
  \end{pmatrix} \; A^{(1)} = T_{31}^{T}AT_{31} = \begin{pmatrix}
    2.11403 & -1.02557 & 0\\
    -1.02557 & 2.06374 & -0.676153\\
    0 & -0.676153 & 4.82223
  \end{pmatrix}
\]\[
  \epsilon > \sum_{i>j}a_{ij}^{2} = 1.51
\]
Оптимальный элемент --- $a_{21} = -1.02557$, $c = 0.715, s = -0.698$ \[
  T_{21} = \begin{pmatrix}
    0.715 & -0.698 & 0\\
    0.698 & 0.715 & 0\\
    0 & 0 & 1
    \end{pmatrix} \; A^{(2)} = T_{21}^{T}A^{(1)}T_{21} = \begin{pmatrix}
    3.11476 & 0 & 0.472217\\
    0 & 1.06301 & -0.483936\\
    0.472217 & -0.483936 & 4.82223
  \end{pmatrix}
\]\[
  \epsilon > \sum_{i>j}a_{ij}^{2} = 0.4571
\]\[
  A^{(3)} = T_{32}^{T}A^{(2)}T_{32} = \begin{pmatrix}
    3.11476 & 0.0593404 & 0.468474\\
    0.0593404 & 1.00171 & 0\\
    0.468474 & 0 & 4.88353
  \end{pmatrix}
\;
\epsilon > \sum_{i>j}a_{ij}^{2} = 0.223
\]\[
  A^{(4)} = T_{31}^{T}A^{(3)}T_{31} = \begin{pmatrix}
    2.99834 & 0.05759 & 0\\
    0.05759 & 1.00171 & 0.0143\\
    0 & 0.0143 & 4.99995
\end{pmatrix}
\;
\epsilon > \sum_{i>j}a_{ij}^{2} = 3.5\cdot10^{-3}
\]
\section{Подготовка контрольных тестов}
Случайно сгенерирована матрица $A$, удовлетворяющая условиям применимости. Были найдены значения собственных чисел методом Якоби с опимальным выбором ключевого элемента
с точностью $\epsilon$ от $10^{-2}$ до $10^{-14}$ и собственные векторы методом обратных итераций с такой же точностью.

Получены значения и построены графики ошибки при нахождении собственных чисел $||\Lambda - \Lambda^{*}||$,
ошибки при нахождении собственного вектора, соотв. минимальному с.ч. $\lambda_{min}$ $||x-x^{*}||$, норма невязки $||Ax - \lambda_{min} x||$ и число итераций в зависимости от задаваемой точности.
\section{Модульная структура программы}
Определены арифметические операции с матрицами, умножение матрицы на число и умножение матриц.\\
$norm(v)$ --- вторая норма вектора\\
$optEl(A)$ --- возвращает координаты оптимального элемента принимаемой матрицы $A$\\
$sumBotLeft(A)$ --- возвращает сумму квадратов элементов под главной диагональю принимаемой матрицы $A$\\
$rotTransform(A, i,j)$ --- принимает матрицу $A$ и пару целых чисел $i,j$, возвращает матрицу, над которой было произведено преобразование вращения, обнуляющее элемент по координатам $i,j$ матрицы $A$\\
$condGen(size, cond)$ --- принимает размер $size$ и число $cond$, генерация матрицы заданного размера, с отделимостью собственных чисел равной $cond^{1/size}$, возвращается сама матрица и вектор собственных чисел\\
$JacobiMethod(A, eps)$ --- нахождение собственных чисел матрицы $A$ методом Якоби с точностью $eps$, возвращается кол-во итераций и вектор собственных чисел.\\
$inverseIteration(A,l,eps)$ --- нахождение собственного вектора матрицы $A$ методом обратных итераций соответствующего с.ч. $l$ с точностью $eps$.
$solveAb(A,b)$ --- решение СЛАУ задаваемой матрицей $A$ и столбцом $b$ методом отражений, возвращается решение.
\pagebreak
\section{Анализ результатов}
\begin{figure}[h]
\includegraphics[width = 0.5\textwidth]{precision.jpg}
\includegraphics[width = 0.5\textwidth]{iterations.jpg}
\end{figure}

Из графиков видно, что количество итераций метода Якоби зависит от требуемой точности нелинейно и количество итераций, требуемых для уточнения на порядок уменьшается с повышением исходной точности.
Количество итераций метода обратных итераций от требуемой точности практически не зависит.

При малой требуемой точности точность может не достигаться, однако при более строгом условии точность достигается почти всегда.

Точность вычисления вектора собственных значений зависит только от количества итераций метода обратных итераций.
Невязка зависит от обоих погрешностей, но ближе к погрешности собственного числа.

При заданной точности $\epsilon = 10^{-11}$, ошибка с.ч. составила $||\Lambda - \Lambda^{*}|| = 4.44 \cdot 10^{-12}$, ошибка с.в. составила $||x - x^{*}|| = 1.07 \cdot 10^{-15}$, невязка составила $||Ax - \lambda x|| = 7.10\cdot10^{-13}$, точность достигнута для всех искомых значений.
\section{Вывод}
Методу Якоби требуется довольно много итераций для нахождения изначального значения, но уточнение значения происходит быстро.
Иногда точность не достигается, что связано скорее всего с тем, что используемый критерий не связан напрямую с оценкой собственных чисел. Оценивать точность по невязке сложно, так как на невязку влияет как точность с.ч., так и точность с.в.
\end{document}
