\documentclass[a4paper]{article}
\usepackage[T1,T2A]{fontenc}
\usepackage[russian]{babel}
\usepackage{amsmath, amsfonts, amssymb}
\usepackage[lmargin=5mm, rmargin=5mm]{geometry}
\usepackage{emptypage}
\usepackage[varbb]{newpxmath}
\usepackage{mathtools}

\newcommand{\dd}[1][]{\ensuremath{\,\mathrm{d}^{#1}}}
\DeclarePairedDelimiter{\abs}{\lvert}{\rvert}

\begin{document}
\begin{titlepage}
  \Large
  \begin{center}
    Санкт-Петербургский\\
    Политехнический университет Петра Великого\\
    \vspace{10em}
    Отчёт по лабораторной работе №7\\
    \vspace{2em}
    \textbf{Решение краевой задачи для ОДУ 2-го порядка}
  \end{center}
  \vspace{6em}
  \begin{flushright}
    Студент: Копнов Александр Александрович\\
    Преподаватель: Добрецова Светлана Борисовна\\
    Группа: 5030102/00003
  \end{flushright}
  \vspace{\fill}
  \begin{center}
    Санкт-Петербург\\
    2022
  \end{center}
\end{titlepage}
\pagebreak

\section{Формулировка задачи}\label{sec:S1}
Задача: найти численное решение краевой задачи на равномерной сетке с помощью метода конечных разностей второго порядка.

Исследовать зависимость фактической ошибки от длины шага и от возмущения начальных условий при фиксированном шаге.

\subsection{Формализация:}\label{subsec:SS1}
Задана линейная двухточечная краевая задача для обыкновенного дифференциального ур-я второго порядка.
\begin{equation}
\label{eq:1}
  L[y] = y'' + p(x)y' + q(x)y = f(x)
\end{equation}
\[
  l_{a}[y] = \alpha_{0}y(a) + \alpha_{1}y'(a) = A
\]
\[
  l_{b}[y] = \beta_{0}y(b) + \beta_{1}y'(b) = B
\]
где
\[
  \alpha_{0}^{2}+\alpha_{1}^{2} \neq 0 \quad
  \beta_{0}^{2}+\beta_{1}^{2} \neq 0
\]
а функции \(p(x),q(x),f(x)\) должны быть такими, чтобы задача имела единственное решение \(y=y(x)\)

\section{Алгоритм}\label{sec:S2}
\textbf{Метод конечных разностей второго порядка}.
Идея метода заключаетсв в том, что вместо производных в дифференциальном уравнении используются их конечноразностные
приближения. Это приводит к необходимости решения алгебраической системы, которая обычно решается методом прогонки.

Вводим на отрезке \([a,b]\) сетку с шагом \(h = \frac{b-a}{n}\) \[
  x_{k} = a + kh, k = \overline{0,n}
\]
На этой сетке определяются сеточные функции \[
  p_{k} = p(x_{k}) \quad q_{k} = q(x_{k}) \quad f_{k} = f(x_{k})
\]
отвечающие функциональным коэфициентам данного дифференциального уравнения. Считая \(y(x)\) точным решением краевой
задачи, будем обозначать за \(y_{k} \approx y(x_{k})\) \(k\)-ю компоненту искомого каркаса приближенного решения.

Фиксируя в уравнении (\ref{eq:1})  \(x = x_{k}\) с учётом обозначений приходим к равенствам \[
  y''(x_{k}) + p_{k}y'(x_{k}) + q_{k}y(x_{k}) = f_{k} \quad k = \overline{0,n}
\]
В каждом внутреннем узле сетки, то есть при \(k = \overline{1,n-1}\), значения производных аппроксимируем по
симметричным формулам второго порядка точности. \[
  y'(x_{k}) = \frac{y(x_{k+1})-y(x_{k-1})}{2h}+O(h^{2})
\]\[
  y''(x_{k}) = \frac{y(x_{k+1})-2y(x_{k})+y(x_{k-1})}{h^{2}} + O(h^{2})
\]

Подставив в равенства, получаем \[
  \frac{y(x_{k+1})-2y(x_{k})+y(x_{k-1})}{h^{2}} + p_{k}\frac{y(x_{k+1})-y(x_{k-1})}{2h} + q_{k}y(x_{k}) = f_{k} + O(h^{2})
\]
Отбросив \(O(h^{2})\) и приведя подобные, получаем
\[
  \left( 1 + \frac{h}{2}p_{k} \right)y_{k+1} - (2- h^{2}q_{k})y_{k} + \left(1-\frac{h}{2} p_{k}\right)y_{k-1} = h^{2}f_{k}
\]
где \(k = \overline{1,n-1}\)

Данные уравнения являются компактной записью СЛАУ с трёхдиагональной матрицей коэфициентов, в которой \(n-1\) уравнение
и \(n+1\) неизвестная.

Два недостающих уравнения получаются на основе краевых условий задачи. Для метода второго порядка используется
аппроксимация второго порядка точности \[
y'(a) = \frac{-3y(x_{0})+4y(x_{1})-y(x_{2})}{2h} + O(h^{2})
\]\[
  y'(b) = \frac{y(x_{n-2})-4y(x_{n-1})+3y(x_{n})}{2h} + O(h^2)
\]
Имеем \[
  \alpha_{0} y\left(x_{0}\right)+\alpha_{1} \frac{-3 y\left(x_{0}\right)+4 y\left(x_{1}\right)-y\left(x_{2}\right)}{2 h}+O\left(h^{2}\right)=A
\]
\[
  \beta_{0} y\left(x_{n}\right)+\beta_{1} \frac{y\left(x_{n-2}\right)-4 y\left(x_{n-1}\right)+3 y\left(x_{n}\right)}{2 h}+O\left(h^{2}\right)=B
\]
Откуда следуют дополнительные связи между тремя первыми и тремя последними переменными. \[
  \left(2 h \alpha_{0}-3 \alpha_{1}\right) y_{0}+4 \alpha_{1} y_{1}-\alpha_{1} y_{2}=2 A h
\]\[
  \beta_{1} y_{n-2}-4 \beta_{1} y_{n-1}+\left(2 h \beta_{0}+3 \beta_{1}\right) y_{n}=2 B h
\]
Для создания структуры коэфициентов, соответствующей методу прогонки, нужно с помощью уравнений при \(k=1; k=n-1\)
исключить \(y_{2}\) и \(y_{n-2}\). Подставив, получаем такие выражения:
\[
  \left(2h\alpha_{0} - \alpha_{1}\left(3 - \frac{2-hp_{1}}{2+hp_{1}}\right)\right)y_0 + \left(4 - \left(\frac{4-2h^2q_{1}}{2+hp_{1}}\right)\right)\alpha_{1}y_1 =
  2Ah + \alpha_{1}\left( \frac{2h^{2}f_{1}}{2+hp_{1}} \right)
\]
\[
  \left(\frac{4-2h^{2}q_{n-1}}{2-hp_{n-1}} - 4\right)\beta_{1}y_{n-1} + \left( 2h\beta_{0} +\beta_{1}\left(3 - \frac{2+hp_{n-1}}{2-hp_{n-1}}\right)\right)y_{n} =
  2Bh - \beta_{1}\left( \frac{2h^{2}f_{n-1}}{2-hp_{n-1}} \right)
\]
Полученная система затем решается методом прогонки, столбец решений является каркасом приближенного решения.

\section{Предварительный анализ задачи}\label{sec:S3}
Задача \[
  xy'' + y' +2y = 2\ln x, x \in [1,2]
\]\[
  1\cdot\ln 1 + 1\cdot \frac{1}{1} = 1
\]\[
  1 \cdot \ln 2 + 0 \cdot \frac{1}{2} = \ln 2
\]
имеет единственное решение \(y = \ln x\), и является корректной.
\section{Тестовый пример}\label{sec:S5}
Найдём решение данной краевой задачи, используя метод конечных разностей второго порядка и сетку \[
  x_{0} = 1 \quad x_{1} = 1.25 \quad x_{2} = 1.5 \quad x_{3} = 1.75 \quad x_{4} = 2
\]\[
  p(x) = \frac{1}{x} \quad q(x) = \frac{2}{x} \quad f(x) = \frac{2\ln x}{x}
\]
Из граничных условий:
\[
  p_1 = 0.8 \quad q_1 = 1.6 \quad f_1 = 0.357
\]\[
  p_3 = 0.571 \quad q_3 = 1.142 \quad f_3 = 0.639
\]\[
  (\frac{2}{4}\cdot 1 - \left( 3 - \frac{2-\frac{p_{1}}{4}}{2+\frac{p_1}{4}} \right))y_{0} + \left( 4 - \frac{4-\frac{q_1}{8}}{2+\frac{p_1}{4}} \right)y_{1} = \frac{1}{2} + \frac{\frac{f_1}{8}}{2+\frac{p_1}{4}}
\]\[
  0 \cdot y_{3} + \frac{1}{2} y_{4} = \frac{\ln 2}{2}
\]
СЛАУ: \[
  \begin{pmatrix}
    0 & -1.682 & 2.273  \\
    0.9 & -1.9 & 1.1 \\
    0.917 & -1.917 & 1.083  \\
    0.928 & -1.929 & 1.071  \\
    0 & 0.5 & 0
  \end{pmatrix}
  \begin{pmatrix}
    y_{0} \\ y_{1} \\ y_{2} \\ y_{3} \\ y_{4}
  \end{pmatrix} =
  \begin{pmatrix}
    0.520 \\0.022\\0.0338\\0.0400\\0.346
  \end{pmatrix}
\]
решая СЛАУ получим \[
  \begin{pmatrix}
    0.250056\\
    0.413967\\
    0.530729\\
    0.619891\\
    0.693147\\
  \end{pmatrix}
\]
Точные значения в данных точках: \[
  \begin{pmatrix}
    0 \\ 0.223 \\ 0.405 \\ 0.559 \\ 0.693
  \end{pmatrix}
\]
Погрешность примерно равна \(0.25 = h = h^{2}\cdot4\).
\section{Подготовка контрольных тестов}\label{sec:S6}
Было найдено решение краевой задачи на отрезке \([1,2]\) с начальным шагом \(0.1\), \(10\) раз уменьшаемым в два раза.
Для первых двух итераций сохранены все значения на отрезке, для остальных бесконечная норма вектора ошибки (максимальная
ошибка на отрезке).

Также была исследована ошибка при возмущении граничных условий от \(\Delta = 10^{-13}\) до \(10^{-1}\) при шаге \(3.125\cdot10^{-3}\)
\section{Модульная структура программы}\label{sec:S7}
\(coefMat(a,b,bounds,n,p,q,f)\) --- возвращает расширенную матрицу \((n+1) \times 4\) коэфициентов СЛАУ, соответствующую
решению линейной краевой задачи с коэфициентами в матрице \(2 \times 3\) \(bounds\) и функциями коэфициентов при \(y\) \(p,q,f\).
\(ThomasRL(B)\) --- решение расширенной матрицы \(B\) методом прогонки справа налево.
\section{Анализ результатов}\label{sec:S8}
Ошибка пропорциональна квадрату длины шага (\(err \approx 4\cdot h^{2}\)).\\
При возмущении входных данных наблюдается минимум ошибки при \(\Delta = +1.4\cdot10^{-5}\).

\section{Вывод}\label{sec:S9}
Практические результаты совпадают с теорией, меньшая ошибка при неточных входных данных вызвана тем, что возмущения
компенсируют погрешность метода.

Погрешность тестового примера совпала с результатами полного исследования.
\end{document}
