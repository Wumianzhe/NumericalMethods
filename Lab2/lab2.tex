\documentclass[a4paper]{article}
\usepackage[T1,T2A]{fontenc}
\usepackage[russian]{babel}
\usepackage{amsmath, amsfonts, amssymb}
\usepackage[lmargin=5mm, rmargin=5mm]{geometry}
\usepackage{emptypage}

\begin{document}
\begin{titlepage}
  \Large
  \begin{center}
    Санкт-Петербургский\\
    Политехнический университет Петра Великого\\
    \vspace{10em}
    Отчёт по лабораторной работе №2\\
    \vspace{2em}
    \textbf{Приближение табличных функций сплайнами}
  \end{center}
  \vspace{6em}
  \begin{flushright}
    Студент: Копнов Александр Александрович\\
    Преподаватель: Добрецова Светлана Борисовна\\
    Группа: 5030102/00003
  \end{flushright}
  \vspace{\fill}
  \begin{center}
    Санкт-Петербург\\
    2022
  \end{center}
\end{titlepage}
\pagebreak

\section{Формулировка задачи}
Задача: построить интерполяционный сплайн для гладкой функции $f(x) = \ctg x + x^{2}$ и функции с разрывом производной $g(x) = f(1.7) + |f(x) - f(1.7)|$ используя равномерную сетку и условие на первую производную в граничных точках. Исследовать влияние числа узлов и гладкости функции на качество интерполяции.
\subsection{Формализация}
Дана сетка $x^{h} := \{x_{i}\}_{i=0}^{n}$ и сеточная функция $y^{h}:=\{y_{i}\}_{i=0}^{n}$\\
Найти сплайн \[
  S = \left\{
\begin{array}{l@{\quad:\quad}l@{\quad}l}
 P_{i}(x)& P_{i}(x_{i}) = y_{i}, P_{i}(x_{i-1}) = y_{i-1} & i \in \overline{1,n}; x \in [x_{i-1};x_{i}]
\end{array}\right.
\]
\section{Алгоритмы}
\textbf{Кубический интерполяционный сплайн}
Необходимо найти \(4n\) коэффициентов для \(n\) функций вида \[
  g_{i} = a_{i}x^{3} + b_{i}x^{2} + c_{i}x + d_{i} \quad x \in [x_{i-1},x_{i}], \quad  i = 1, \ldots , n
\]
Из формулировки задачи имеются условия непрерывности до второй производной для
внутренних точек \[
  \begin{cases}
    g_{i}(x_{i}) = g_{i+1}(x_{i})\\
    g'_{i}(x_{i}) = g'_{i+1}(x_{i})\\
    g''_{i}(x_{i}) = g''_{i+1}(x_{i})
  \end{cases}
\]
Также имеется условие интерполирования \(g_{1}(x_{0}) = y_{0}\) и
\(g_{i}(x_{i}) = y_{i}\) для всех \(i = 1,\ldots,n\). Суммарно имеется
\(3(n-1)+n+1=4n-2\) условий. Добавив 2 граничных условия
\(g'_{1}(x_{0}) = y'(x_{0})\) и \(g'_{n}(x_{n}) = y'(x_{n})\) мы можем решить
систему уравнений и найти все функции.

Так как \(g\) --- полином третьей степени, \(g''\) --- линейная функция. Так как
\(g''(x_{i}) = M_{i}\), \(g_{i}''(x_{i-1}) = M_{i-1}\) \[
  g''_{i}(x_{i}) = M_{i-1} \frac{x_{i}-x}{h_{i}} + M_{i} \frac{x-x_{i}}{h_{i}} \quad x \in [x_{i-1},x_{i}] \quad h_{i} = x_{i} - x_{i-1}
\]
После интегрирования \[
  g' = M_{i} \frac{(x-x_{i-1})^{2}}{2h_{i}} - M_{i-1} \frac{(x_{i}-x)^{2}}{2h_{i}} + C
\]\[
  g = M_{i-1} \frac{(x_{i}-x)^{3}}{6h_{i}} + M_{i} \frac{(x-x_{i-1})^{3}}{6h_{i}} + C_{i}(x-x_{i-1}) + \tilde{C}_{i}
\]
Константы находятся из условий для границ отрезков по формулам \[
  \tilde{C}_{i} = y_{i-1} - M_{i-1} \frac{h_{i}^{2}}{6}
\]\[
  C_{i} = \frac{y_{i}-y_{i-1}}{h_{i}} - \frac{h}{6} \left(M_{i} - M_{i-1}\right)
\]
Из условия непрерывности первых производных во внутренних узлах следует выражение \[
  \frac{h_{i}}{h_{i}+h_{i+1}} M_{i-1} + 2M_{i} + \frac{h_{i}}{h_{i}+h_{i+1}}M_{i+1} = \frac{6}{h_{i}+h_{i+1}}\left( \frac{y_{i+1}-y_{i}}{h_{i+1}} - \frac{y_{1}-y_{i-1}}{h_{i}}\right) \quad i = 1,\ldots,n-1
\]
\textbf{Условия применимости}
\begin{itemize}
  \item все узлы \(x_{i} (i = 0,\ldots,n)\) попарно различны
  \item сетка \(x^{h}\) является упорядоченной
\end{itemize}
\section{Предварительный анализ задачи}
По теореме Вейерштрасса любую непрерывную функцию можно приблизить полиномом с вещественными коэфициентами. Заданные
функции неперывны как сумма непрерывных
\section{Проверка условий}
Функции \(f\) и \(g\) непрерывны,
Упорядоченная равномерная сетка удовлетворяет условиям применимости по построению
\section{Тестовый пример}
Произведём приближение функции $y = \ctg(x)+x^{2}$ на отрезке $[1;2.5]$ на сетке из 4 точек:\[
\begin{matrix}
x_{0} = 1& x_{1} = 1.5& x_{2} = 2& x_{3} = 2.5\\
y_{0} = 1.642& y_{1} = 2.321& y_{2} = 3.542& y_{3} = 4.911\\
\end{matrix} \quad \forall i h_{i} = 0.5
\]
Граничные условия: \[
  y'(1) = 0.587 \quad y'(2.5) = 2.208
\]
Будем искать сплайн в виде функции \(g\) составленной из функций \(g_{i}\), определённых на последовательно взятых
частях отрезка интерполирования.

Замена. \(g''(x_{i}) = M_{i}, i = 0, \ldots, n\). Из граничных условий получаем \[
  -\frac{h_{1}}{3} M_{0} - \frac{h_{1}}{6}M_{1} = f'(x_{0}) + \frac{y_{0}-y_{1}}{h_{1}} \rightarrow -\frac{1}{6} M_{0} - \frac{1}{12} M_{1} = -0.770
\]\[
  \frac{h_{n}}{3} M_{n} + \frac{h_{n}}{6} M_{n-1} = f'(x_{n}) + \frac{y_{n-1}-y_{n}}{h_{n}} \rightarrow \frac{1}{6} M_{3} + \frac{1}{12} M_{2} = -0.530
\]
Составим СЛАУ для нахождения значений \(M_{i}\): \[
\begin{pmatrix}
  -\frac{1}{6} & -\frac{1}{12} & 0 & 0\\
  0.5 & 2 & 0.5 & 0\\
  0 & 0.5 & 2 & 0.5 \\
  0 & 0 & \frac{1}{12} & \frac{1}{6} \\
\end{pmatrix} \cdot
\begin{pmatrix}
 M_{0} \\ M_{1} \\ M_{2} \\ M_{3}
\end{pmatrix} = \begin{pmatrix}
  -0.770\\ 6.511 \\ 1.771 \\ -0.530
\end{pmatrix}
\]
Решая СЛАУ получим \[
\begin{pmatrix}
 M_{0} \\ M_{1} \\ M_{2} \\ M_{3}
\end{pmatrix} = \begin{pmatrix}
  3.611 \\ 2.016 \\ 1.344 \\ -3.852
\end{pmatrix}
\] Тогда константы равны: \[
  C: \begin{pmatrix}
    1.491 \\ 2.237 \\ 3.486
  \end{pmatrix} \\ \tilde{C} \begin{pmatrix}
    1.490 \\ 2.499 \\ 3.171
  \end{pmatrix}
\]
Соответственно функции: \[
  \begin{matrix}
    g_{1} = &-0.531x^{3} &+ 3.400x^{2} &-4.618x &+3.389\\
    g_{2} = &-0.224x^{3} &+ 2.016x^{2} &-2.803x &+3.007\\
    g_{3} = &-1.732x^{3} &+11.064x^{2} &-20.322x&+13.471\\
  \end{matrix}
\]
Первая из полученных функций близка к графику табличной функции, последующие же имеют ошибку порядка \(10^{0}\), что
вызвано, скорее всего, округлением до трёх знаков после запятой.
\section{Подготовка контрольных тестов}
Были построены приближения функций $f(x)$ и $g(x)$.\\
Приближение строилось на отрезке $[1;2.5]$. Генерировались сетки заданного размера и значение между узлами сравнивалось со значением исходной функции.\\
По полученным результатам построены графики приближений сплайнов из 3,5 и 7 функций, графики ошибки для этих же сплайнов
и график максимальной ошибки на промежутке в зависимости от количества точек в сетке.
\section{Модульная структура программы}
$meshGen(a,b,n,f)$ --- генерация равномерной сетки на отрезке $[a,b]$ в которой $n+1$ узел и значения сеточной функции равны значениям функциии $f$. Возвращает сгенерированную сетку\\
\(coefMat(mesh, df)\) --- генерация матрицы коэфициентов по сетке \(mesh\) и функции производной исходной функции \(df\).
Возвращается расширенная матрица СЛАУ.\\
\(ThomasAlg(B)\) --- решение СЛАУ из расширенной матрицы \(B\) методом прогонки. Возвращает столбец решений.\\
\(constsMat(mesh, M)\) --- по сетке \(mesh\) и столбцу значений второй производной \(M\) вычисляются все константы \(C, \tilde{C}\).
Возвращается матрица коэфициентов из трёх столбцов со значениями второй производной и константами \(C, \tilde{C}\).\\
\(splineApprox(CM,mesh,x)\) --- по сетке \(mesh\), значению аргумента \(x\) и матрице коэфициентов \(CM\) возвращает
значение сплайна в точке.
\section{Анализ результатов}
Для гладкой функции при увеличении числа точек точность увеличивается монотонно, увеличение размера матрицы в два раза
увеличивает точность чуть более чем на порядок. Вероятно возможно достигнуть машинной
точности путём увеличения разрешения сетки и размера матрицы, но это также увеличивает затраты на вычисления. Зависимость от
граничных условий значительна, использование точных значений уменьшает ошибку более чем на порядок.

Для функции с разрывом первой производной точность увеличивалась медленно, большое влияние колебаний в ошибке с малым
периодом. Зависимость ошибки от граничных условий практически отсутствует.
\section{Вывод}
Колебания точности в функции с разрывом первой производной скорее всего, подобно первой работе, зависят от расположения
точки разрыва внутри сетки.

Сплайны позволяют достичь теоритически большей точности за счёт стабильности задачи, но для малых размеров сетки
достигаемая сплайнами точность значительно меньше, чем для интерполяции полиномом. Также сплайны лучше подходят для
интерполяции функции с разрывом в производной.
\end{document}
