\documentclass[11pt,a4paper]{article}
\usepackage[T1,T2A]{fontenc}
\usepackage[russian]{babel}
\usepackage{amsmath, amsfonts, amssymb}
\usepackage[lmargin=5mm, rmargin=5mm]{geometry}
\usepackage{emptypage}

% epsilon is varepsion
\let\epsilon\varepsilon

\begin{document}
\begin{titlepage}
  \Large
  \begin{center}
    Санкт-Петербургский\\
    Политехнический университет Петра Великого\\
    \vspace{10em}
    Отчёт по лабораторной работе №2\\
    \vspace{2em}
    \textbf{Решение Систем Линейных Алгебраических Уравнений\\
    прямыми методами}
  \end{center}
  \vspace{6em}
  \begin{flushright}
    Студент: Копнов Александр Александрович\\
    Преподаватель: Добрецова Светлана Борисовна\\
    Группа: 5030102/00003
  \end{flushright}
  \vspace{\fill}
  \begin{center}
    Санкт-Петербург\\
    2021
  \end{center}
\end{titlepage}
\tableofcontents
\pagebreak

\section{Формулировка задачи и её формализация}
\textbf{Задача:} найти решение $X$ системы линейных алгебраических уравнений $AX = B$ методом отражений. Исследовать зависимость ошибки и невязки от числа обусловленности матрицы.\\
\textbf{Формализация:}\\
Дана СЛАУ $AX = B$, где $A \in M_{n \times n}$, $B \in M_{n \times 1}$\\
Если $det(A) \neq 0 \Rightarrow \exists! x^{*} \in M_{n\times1} : Ax^{*} = B$\\
Точный метод находит решение за конечное число шагов без возможности задать погрешность.
\section{Алгоритм}
Алгоритм делится на две подзадачи: преобразование матрицы методом отражений, с помощью которого матрица приводится к верхнетреугольному виду и обратный ход методом Гаусса, с помощью которого последовательно выражаются переменные.\\
\textbf{Метод отражений:}\\
Пусть имеется СЛАУ заданная расширенной матрицей $B = (A|b)$. Одинаковыми преобразованиями, сохраняющими эквивалентность системы, матрицу $A$ нужно привести к верхнетреугольной матрице $R$, а вектор $b$ к вектору $y = Q^{T}b$, где
$A = QR$ и $QRx = B$, $Q$ --- ортогональная. Иными словами, надо привести расширенную матрицу $B$ в расширенную матрицу $C = (R|y)$. Это можно сделать, применяя последовательно к столбцам матрицы $B$ преобразования Хаусхолдера.
\[
  R = H_{n-1}\ldots H_{2}H_{1}A \; Q^{T} = H_{n-1}\ldots H_{2}H_{1} \Rightarrow Q^{T}B = C \; B_{i}=H_{i}B_{i-1} \; i = 1,2,\ldots ,n-1
\]
где $H_{i}$ --- матрица Хаусхолдера $i$-го этапа, определяемая формулами \[
  H_{i} = E - 2w_{i}w_{i}^{T} \; w_{i} = \mu_{i}(0;\ldots;a_{i,i}^{(i-1)}-\beta_{i};a_{i+1,i}^{(i-1)};\ldots;a_{n,i}^{(i-1)}) \; \beta_{i} = - sign(a_{i,i}^{(i-1)}) \cdot \sqrt{\sum_{k=i}^{n}(a_{ki}^{(i-1)})^{2}}, \mu_{i} = \frac{1}{2\beta_{i}^{2}-2\beta_{i}a_{i,i}^{(i-1)}}
\]
После $n-1$ таких преобразований получаем СЛАУ с верхнетреугольной матрицей коэфициентов \[ \left(
  \begin{array}{cccc|c}
    a_{11}^{(n-1)} & a_{12}^{(n-1)} & \cdots & a_{1n}^{(n-1)} & b_{1}^{(n-1)}\\
    0 & a_{22}^{(n-1)} & \cdots & a_{2n}^{(n-1)} & b_{2}^{(n-1)}\\
    \vdots & \vdots & \ddots & \vdots & \vdots\\
    0 & 0 & \cdots & a_{nn}^{(n-1)} & b_{n}^{(n-1)}\\

  \end{array} \right) = \left(
  \begin{array}{cccc|c}
    a_{11}^{(1)} & a_{12}^{(1)} & \cdots & a_{1n}^{(1)} & b_{1}^{(1)}\\
    0 & a_{22}^{(2)} & \cdots & a_{2n}^{(2)} & b_{2}^{(2)}\\
    \vdots & \vdots & \ddots & \vdots & \vdots\\
    0 & 0 & \cdots & a_{nn}^{(n-1)} & b_{n}^{(n-1)}\\

  \end{array} \right)
\]
\textbf{Обратный ход методом Гаусса:}\\
\[
  x_{k} = \frac{1}{a_{kk}^{(n-1)}}\left( b_{k}^{(n-1)} - \sum_{j=k+1}^{n}a_{kj}^{(n-1)}x_{j} \right) \; k = n, \ldots, 2, 1
\]
Где $x_{k}$ --- $k$-я компонента искомого вектора $X$
\section{Предварительный анализ задачи}
Матрица коэфициентов СЛАУ не является вырожденной, т.е. $\det{A}\neq 0$. Матрица $A$ строится как $A = Q_{1}DQ_{2}$, где матрица $D$ --- диагональная не вырожденная, $Q_{1},Q_{2}$ --- ортогональные.
\section{Проверка условий}
Метод отражений не требует структурных особенностей матрицы, следовательно единственное условие это невырожденность и оно выполняется автоматически по построению.
\section{Тестовый пример}
Решим методом отражений систему \[ \left\{
\begin{array}{lllc}
  x_{1} &- 2x_{2} &+ x_{3} &= 1\\
  2x_{1} & & -3x_{3} &= 8\\
  2x_{1} &-x_{2} & -x_{3} &= 5
\end{array} \right.
\]
Для этого нужно выполнить два этапа преобразования Хаусхолдера над расширенной матрицей \[
  B = (A|b) = \left(
    \begin{array}{ccc|c}
      1 & -2 & 1 & 1\\
      2 & 0 & -3 & 8\\
      2 & -1 & -1& 5
    \end{array} \right)
\]
а затем сделать обратный ход методом Гаусса.\\
На первом этапе имеем \[
  \beta_{1} = -sign(a_{11})\cdot\sqrt{a_{11}^{2}+a_{21}^{2}+a_{31}^{2}} = - \sqrt{1^{2} + 2^{2} + 2^{2}} = -3 \quad
  \mu_{1} = \frac{1}{2\beta_{i}^{2} - 2\beta_{1}a_{11}} = \frac{1}{2(-3)^{2} - 2(-3)1} = \frac{1}{2\sqrt{6}}
\]\[
  w = \frac{1}{2\sqrt{6}}\begin{pmatrix}
    1 + 3\\ 2 \\ 2
  \end{pmatrix} \; H_{1} = \frac{1}{3} \begin{pmatrix}
    -1 & -2 & -2\\
    -2 &  2 & -1\\
    -2 & -1 & 2
  \end{pmatrix} \; H_{1}B = \frac{1}{3} \left(
    \begin{array}{ccc|c}
      -9 & 4 & 7 & -27\\
      0 & 5 & -7& 9\\
      0 & 2 & -1& 0
    \end{array}
  \right)
\]\[
  \beta_{2} = -sign(5/3)\frac{1}{3}\sqrt{5^{2}+ 2^{2}} = -1.795 \quad \mu_{2} = \frac{1}{2\beta^{2}_{2} - 2\beta_{2}a_{22}} = 0.284
\]\[
  w = 0.284\begin{pmatrix}
    0 \\ \frac{5}{3} + 1.795 \\ 2
\end{pmatrix} \; H_{2} = \begin{pmatrix}
  1 & 0 & 0\\
  0 & -0.932 & -0.371 \\
  0 & -0.371 & 0.928
\end{pmatrix} \quad C = H_{2}B_{1} = \left(
    \begin{array}{ccc|c}
      -3 & \frac{4}{3} & \frac{7}{3} & -9\\
      0 & -1.795 & 2.290& -2.785\\
      0 & 0 & 0.557& -1.114
    \end{array}
  \right)
\]
При помощи обратного хода метода Гаусса получаем решение системы \[
  x_{3} = \frac{-1.114}{0.557} = -2
\]\[
  x_{2} = \frac{-2.785 - 2.290\cdot(-2)}{-1.795} = -1
\]\[
  x_{1} = \frac{-9 - 1.33\cdot(-1) - 2.333(-2)}{-3} = 1
\]
\section{Подготовка контрольных тестов}
Было случайно сгенерировано точное решение $X^{*}$, затем сгенерированы матрицы коэфициентов с заданным числом обусловленности. Левая часть системы была получена как $b = AX^{*}$. Каждая СЛАУ была решена методом отражений.
По полученным результатам были построены графики зависимости ошибки $||x-x^{*}||$ и невязки $||Ax - b||$ от числа обусловленности использованного при генерации.

Были сгенерированы точное решение и матрицы коэфициентов с ``хорошим'' и ``плохим'' числом обусловленности (10 и 10000) и получены левые части.
Затем был сгенерирован столбец возмущения, умноженый на такой коэфициент, что $ \frac{||\delta b||}{||b||} = dist$, где $dist$ изменяется от $10^{-7}$ до $0.1$, на каждом шаге увеличиваясь в 1.5 раза.
СЛАУ $AX = \tilde{b} = b + db$ была решена методом отражений. Были построены графики зависимости относительной ошибки $\frac{||x-x^{*}||}{||x^{*}||}$ от возмущения $\frac{||\delta b||}{||b||}$.

\section{Модульная структура программы}
Определены арифметические операции с матрицами, умножение матрицы на число и умножение матриц.\\
$norm(v)$ --- вторая норма вектора\\
$reflectionTransform(B)$ --- решение СЛАУ методом отражений с обратным ходом методом Гаусса. Функция принимает расширенную матрицу системы и возвращает найденное решение.\\
$condGen(cond,size)$ --- генерация матрицы с заданным размером и числом обусловленности.\\
$solveAb(A,b)$ --- функция принимает отдельно правую и левую части системы, возвращает решение, полученное из $reflectionTransform$\\
$reflectionVarCond(cond, X)$ --- генерация и решение системы с заданным числом обусловленности и точным решением. Возвращает нормы ошибки и невязки\\

\section{Анализ результатов}
Из графика видно, что при увеличении числа обусловленности на порядок, невязка возрастает примерно на пол порядка, а ошибка примерно на порядок, но имеет большой разброс.
Это совпадает с данными в зависимости от возмущения, числа обусловленности отличаются на 3 порядка, а ошибка отличается также примерно на 3 порядка.
При увеличении ошибки входных данных на порядок, фактическая ошибка также увеличивается примерно на порядок.
\section{Вывод}
Как для ``хорошей'', так и для ``плохой'' матрицы всё время выполняется неравенство $\frac{||y-x||}{||x||} \leq ||A||\cdot||A^{-1}|| \frac{||\delta b||}{||b||}$.
Например, при возмущении $3.84\cdot10^{-6}$ относительная ошибка для ``хорошей'' матрицы $5.8\cdot10^{-6}$, для ``плохой'' матрицы --- $4.49\cdot10^{-3}$.
Соответственно неравенства имеют вид $5.8\cdot10^{-6} \leq 10 \cdot 3.84\cdot10^{-6}$ и $4.49\cdot10^{-3} \leq 10^{4} \cdot3.84\cdot10^{-6}$\\
При малых значениях числа обусловленности ошибка почти не отличается от невязки, значит невязка является хорошим способом определения ошибки. Однако, при большом числе обусловленности фактическая ошибка может быть на несколько порядков больше невязки, что затрудняет оценку ошибки.
\end{document}
